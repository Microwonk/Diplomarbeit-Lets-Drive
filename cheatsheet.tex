% !TeX root = ./Latex Vorlage HAK Imst.tex
\chapter{Beispielkapitel}
\section{Beispiele zitieren}

Das ist ein Zitat mit Klammern,
\citep{resnick_distributed_1996}, das ein Zitat ohne Klammern:
\cite{harel_situating_1991}. Hier das selbe Zitat mit einer Seitenangabe und Klammern \citep[S. 23]{resnick_distributed_1996}.

Wird ein Absatz aus einer Quelle sinngemäß übernommen (nicht wörtlich), dann kann nach dem Absatz das entsprechende Zitat in Klammern angeführt werden. \citep[S. 33]{anastopoulou_constructionism_2012}

Wenn ein Zitat im Text angegeben wird, wie z.B. so \cite{beer_rudolf_aspekte_2011}, können die Klammern weggelassen werden.

Der folgende Absatz zeigt ein Blockzitat (wörtlich übernommene Textpassage aus einer Quelle):

\blockcquote[S. 21]{ackermann_piagets_2001}{
	Dr. Heinrich Faust ist ein angesehener Wissenschaftler und Akademiker, der trotz seiner wissenschaftlichen Studien und einer guten Bildung seinen Wissensdurst nicht stillen kann. Eines Nachts sitzt er in seinem Studierzimmer und grübelt über den Sinn des Lebens nach, findet jedoch keine Antworten.
	Daraufhin wendet er sich der Geisterwelt zu. Er beschwört einen Erdgeist, versucht sich den Geistern gleich zu stellen, was ihm jedoch nicht gelingt. Von Ohnmacht getrieben will er sich das Leben nehmen. Sein Selbstmordversuch wird jedoch von Glockenläuten zum Ostertag und seinen Kindheitserinnerungen gestört.
}

Hier wird ein wörtliches Zitat inline angegeben: \blockcquote{gohlich_lernen:_2007}{Das ist ein kleines direktes Zitat.}, und danach geht es gleich wieder direkt weiter. Ob ein wörtliches Zitat inline oder als eigener Block angezeigt wird, entscheidet \LaTeX{} auf Basis der Länge.

%So kann man den Author umdefinieren, der für den Seiteninhalt
%verantwortlich ist
\def \currentAuthor {Harald Sohm}

\section{Beispiele Abbildungen}
Auf diese Weise kann man zum Beispiel in \LaTeX{} auf die \cref{fig:ArduExample} verweisen. Die Kennung für den Verweis vergibt
man selbst mit dem "`label"' Kommando bei der Abbildung.

Jede Abbildung muss nicht nur mindestens einen Verweis im Text haben. Es wird außerdem eine Bildunterschrift verlangt. Für diese ist festgesetzt, dass die Abbildungsunterschrift alleine ausreichend sein muss, um zu verstehen, was am Bild zu erkennen ist.

Der nächste wichtige Punkt sind die Quellenangaben bei Abbildungen. Der Author muss zu jeder Abbildung die notwendigen Rechte haben und idealer Weise gibt man diese bei der Abbildung mit an. In \cref{fig:ArduExample} auf Seite \pageref{fig:ArduExample} sieht man das.

Es ist wichtig zu verstehen, dass \LaTeX{} die Positionierung von Abbildungen übernimmt. Man definiert die Abbildung über begin-figure dort, wo man die Abbildung in etwa haben  möchte, den Rest übernimmt \LaTeX{}

\begin{figure}[t]
	\centering
	%\includegraphics[width=0.7\linewidth]{figures/Arduino_board.png}
	\copyrightbox[r]{\includegraphics[width=0.8\linewidth]{figures/Arduino_Example.jpg}}{\textcopyright\
		Stefan Stolz (CC BY-SA 3.0)}
	\caption[Arduino mit Lichtsensor und Lichterkette]{Hintergrund: Arduino Board;
		Vordergrund: eine Lichterreihe und ein Lichtsensor (Fotowiderstand); In diesem
		Beispiel wird die Lichterreiche je nach Helligkeit des Umgebungslichtes
		gesteuert. Durch leichte Modifikationen kann man damit eine Lichtschranke oder
		auch eine Helligkeitssteuerung für das Smartphone simulieren.}
	\label{fig:ArduExample}
\end{figure}

\begin{lstlisting}[style=csharp]
	using System;
	
	
	// kleines Programm
	class Program
	{
		static void Main(string[] args)
		{
			Console.WriteLine("Hello, World!");
		}
	}
\end{lstlisting}

\subsubsection{Beispiele Tabellen}

\cref{tbl:lineBreak} ist ein Beispiel für eine einfache Tabelle mit
Zeilenumbruch, \cref{tbl:shiftReg} für eine aufwändigere Tabelle mit einer
Abbildung und Überschrift.

\begin{table}[h]
	\begin{tabularx}{\textwidth}{|l|X|}
		Use Case & Opret Server                                                                                               \\
		Scenarie & At oprette en server med bestemte regler som tillader folk at spille sammen. More Text more text More Text \\
	\end{tabularx}
	\caption{Einfache Tabelle mit Zeilenumbruch}
	\label{tbl:lineBreak}
\end{table}

\begin{table}
	\begin{center}
		\begin{tabularx}{\textwidth}{ cX  }
			\includegraphics[width=0.3\textwidth]{figures/shift_reg.png} &
			{\begin{tabularx}{\cellwidth}{ lX  }
						$V_{cc}$           & Positive supply voltage                                                              \\
						GND                & Ground                                                                               \\
						SER IN             & Daten Pin                                                                            \\
						SRCK               & Clock Pin                                                                            \\
						RCK                & Latch Pin                                                                            \\
						$\overline{SRCLR}$ & Wenn \textbf{shift-register clear} LOW ist, werden die input Register gelöscht       \\
						$\overline{G}$     & Wenn \textbf{output enable} HIGH ist, werden die Daten im Output Buffer LOW gehalten
					\end{tabularx}   }
		\end{tabularx}
		\caption{Aufwändige Tabelle mit Abbildung und Caption}
		\label{tbl:shiftReg}
	\end{center}
\end{table}

Tabellen sind in \LaTeX{} sehr kompliziert zu erzeugen. Alternativ kann man die
Tabellen auch in einem anderen Programm gestalten und als Bild wieder einfügen.
Dieses Bild kann dann innerhalb von begin-Table verwendet werden. Bedenke aber,
dass die Tabellen in \LaTeX{} zu erzeugen der saubere Weg ist.

\section{Beispiele Listen}
Im Folgenden wird eine Liste gezeigt:
\begin{itemize}
	\item Ich weiß, dass viele Geräte des täglichen Lebens durch Computer
	      gesteuert werden und kann für mich relevante nennen und nutzen.
	      \begin{enumerate}
		      \item Und jetzt eine Numerierung
		      \item Und jetzt eine Numerierung
	      \end{enumerate}
	\item Ich kann wichtige Bestandteile eines Computersystems (Eingabe-,
	      Ausgabegeräte und Zentraleinheit) benennen, kann ihre Funktionen beschreiben
	      und diese bedienen.
\end{itemize}

Und jetzt eine Numerierung:

\begin{enumerate}
	\item Aufzählungspunkt
	      \begin{enumerate}
		      \item Unteraufzählung
		      \item Unteraufzählung
		            \begin{itemize}
			            \item Und jetzt noch eine Ebene ohne Aufzählung
			            \item Und jetzt noch eine Ebene ohne Aufzählung
		            \end{itemize}
	      \end{enumerate}
	\item Aufzählungspunkt
	\item Aufzählungspunkt
	\item Aufzählungspunkt
	\item Aufzählungspunkt
\end{enumerate}

\section{Beispiel Symbole}

Im Internet finden sich umfassende Auflistungen. Hier häufige
Beispiele:

Copyright: \copyright{}, Trademark: \texttrademark{} or \textsuperscript{TM},
Registered: \textregistered{}, DOLLAR: \textdollar{}, EURO: \texteuro{},
Unendlich: \(\infty\) (Mathematik Symbol, deshalb in Mathe Umgebung für Formel), Griechische Symbole: \(\alpha\), \(\beta\), ..., Haken: \(\surd\)

\section{Formeln}

The well known Pythagorean theorem \(x^2 + y^2 = z^2\) was
proved to be invalid for other exponents.
Meaning the next equation has no integer solutions:

\[ x^n + y^n = z^n \]

Im Folgenden ein Bock mit einfacher \cref{eq:simple} und komplexer \cref{eq:complex}, sowie ein Block mit mehreren Zeilen \cref{eq:line1} und \cref{eq:line2}.

\begin{equation}
	\sqrt{a^2 + b^2} = c
	\label{eq:simple}
\end{equation}

\begin{equation}
	r = \dfrac{ \displaystyle\sum\limits_{i=0}^n ((x_i-mx)*(y_{i-d}-my)) } { \sqrt{\displaystyle\sum\limits_{i=0}^n(x_i-mx)^2 }\sqrt{ \displaystyle\sum\limits_{i=0}^n(y_{i-d}-m)^2} }
	\label{eq:complex}
\end{equation}

\begin{align}
	\label{eq:line1} GTCGACGAATTCTAGTTCTAGGGTAAAC \\
	\label{eq:line2} CTTCAATACTACACTTGCAGGATCC
\end{align}




\section{Beispiel Codesequenz}
In \cref{code:qj} sieht man ein Quick-Sort-Listing in der Programmiersprache JAVA. Das Listings-Paket übernimmt die Formatierung von Codebausteinen und kann in der Präambel nach Belieben auf eine andere Sprache konfiguriert werden.

\def \currentAuthor {Author2}

\subsection{Quicksort in JAVA}
\begin{lstlisting}[language=Java, caption=QuickSort in Java, label=code:qj]
public class QuickSort {
public static void main(String[] args) {
int[] x = { 9, 2, 4, 7, 3, 7, 10 };
System.out.println(Arrays.toString(x));

int low = 0;
int high = x.length - 1;

quickSort(x, low, high);
System.out.println(Arrays.toString(x));
}

public static void quickSort(int[] arr, int low, int high) {
if (arr == null || arr.length == 0)
return;

if (low >= high)
return;

// pick the pivot
int middle = low + (high - low) / 2;
int pivot = arr[middle];

// make left < pivot and right > pivot
int i = low, j = high;
while (i <= j) {
while (arr[i] < pivot) {
i++;
}

while (arr[j] > pivot) {
j--;
}

if (i <= j) {
int temp = arr[i];
arr[i] = arr[j];
arr[j] = temp;
i++;
j--;
}
}

// recursively sort two sub parts
if (low < j)
quickSort(arr, low, j);

if (high > i)
quickSort(arr, i, high);
}
}
\end{lstlisting}

\section{Beispieltext}

\Blindtext