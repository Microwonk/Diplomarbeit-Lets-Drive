% !TeX root = ./LetsDriveDipl.tex
\def \currentAuthor {Nicolas Frey} %so kann jederzeit der Autor geändert werden -> wird in der Fusszeile angezeigt.

\chapter*{Einleitende Bemerkungen}



\chapter*{Notationen}
Beschreibung wie Code, Hinweise, Zitate etc. formatiert werden  



\chapter{Einleitung}



\chapter{Projektmanagement}



\section{Metainformationen}
Dieses Kapitel dient der Vorstellung der Schlüsselpersonen und Kooperationspartner, die an der Entstehung dieser Diplomarbeit beteiligt sind. Hier werden das Forschungsteam, die Betreuerin, der Partner und der Ansprechpartner kurz vorgestellt. Die Zusammenarbeit dieser Akteure ist von zentraler Bedeutung für den Erfolg und die Qualität dieser Arbeit.

\subsubsection*{Team}

\emph{Nicolas Frey} \\
\href{mailto:nfrey@tsn.at}{nfrey@tsn.at} 
\begin{figure}
	% include gfx Nicolas Frey Bild
\end{figure}

\emph{Jasmin Meze-Hausken} \\
\href{mailto:jasmeze@tsn.at}{jasmeze@tsn.at} 
\begin{figure}
	% include gfx Jasmin Meze-Hausken Bild
\end{figure}

\emph{Matthias Leiter} \\
\href{mailto:mattleiter@tsn.at}{mattleiter@tsn.at} 
\begin{figure}
	% include gfx Matthias Leiter Bild
\end{figure}

\emph{Anja Holzknecht} \\
\href{mailto:?@tsn.at}{?@tsn.at} 
\begin{figure}
	% include gfx Anja Holzknecht Bild
\end{figure}

\subsubsection{Betreuer}
\emph{Melanie Osl}

\subsubsection{Partner}
\textbf{icotec Graz} \\
\href{mailto:office@icotec.at}{office@icotec.at} \\
\href{www.icotec.at}{www.icotec.at} \\
Andritzer Reichsstraße 15 \\
8045 Graz \\
Tel.: \href{tel:069910624227}{0699 10624227}
% TODO : qualität von logo
\begin{figure}[H]
	\captionsetup{singlelinecheck = false, format= hang, justification=raggedright, font=footnotesize, labelsep=space}
	\includegraphics{Logo_Projektpartner.png}
	\caption{Logo der Firma icotec}
	\label{fig:LogoProjektpartner}
\end{figure}

\subsubsection{Ansprechpartner}
\emph {Ronald Macek}
\begin{figure}
	% include gfx Ronald Macek Bild
\end{figure}


\section{Vorerhebungen}
In den Vorerhebungen werden der IST - Zustand und der SOLL - Zustand genau erklärt. Eine Zielformulierung beschreibt das erwartete Ergebnis eindeutig, attraktiv, realistisch und messbar mit konkreten Terminen. Einfluss, Nähe und Einstellung der Stakeholder und Maßnahmen für diese werden graphisch und tabellarisch dargestellt.

\subsection{IST-Zustand}
Bis dato gibt es keine App, die das Aufzeichnen des Fahrtenbuchs für den Führerschein mit L-Tafel unabhängig von Fahrschulen und Mitgliedschaften (ÖAMTC) ermöglicht. Des Weiteren gibt es kein weitverbreitetes Tool das dem/der Fahrschüler:in personalisiertes Feedback seines/r bisherigen 

\subsection{SOLL-Zustand}
Eine Smartphone-App für Fahrschüler:innen, welche über GPS - Tracking automatisch Einträge in einem Fahrtenbuch erstellt. Diese Einträge werden in Form eines offiziellen Protokolls exportiert. Das Analysieren der Fahrdaten und Erzeugen von Statistiken geben dem/der Fahrschüler:in ein Feedback. Zur Datensicherung steht eine Datenbank zur Verfügung.

\subsection{Projektzieleplan}
Ziel ist die Erstellung einer Smartphone-App für Fahrschüler:innen, welche über GPS - Tracking automatisch Einträge in einem Fahrtenbuch aufzeichnet. Diese Einträge sollen in Form eines offiziellen Protokolls an die Fahrschule ausgehändigt werden können. Um Schüler:innen besser auf die praktische Fahrprüfung vorzubereiten, soll ebenso ein digitaler Fahrlehrer implementiert werden, dem/der Schüler:in Feedback in Form von statistischen Auswertungen gibt. Bis zum 28.6.2024 sollen alle oben genannten Features implementiert, getestet und veröffentlicht werden. Um sich ein Bild vom Markterfolg machen zu können, werden hierbei die App-Store Bewertungen und Downloadzahlen verwendet. Das Ziel ist, bis September die 3\%-Schwelle zu erreichen, das bedeutet etwa 100 Neukunden unter allen Österreicher:innen, die die L-Tafel beantragen (einschließlich L-17).

\subsection{Projektumfeldanalyse}

\subsubsection*{Identifikation der Stakeholder}
Zu Beginn haben wir mittels „Brainstorming“ die wichtigsten Stakeholder identifiziert. Uns wurde ersichtlich, dass nur User also Fahrschüler:innen und Begleitpersonen unsere Zielgruppen sind und mit dem Produkt zufrieden sein müssen. Im Laufe der Recherche wurde eine von ÖAMTC bereitgestellte\hfill\break Konkurrenz-App identifiziert, wodurch uns sofort klar war, dass wir gleich von Beginn an ein starkes Auftreten brauchen. Das bedeutet, dass wir speziell durch bessere Software und Features, sowie durch ein besseres Preismodell attraktiver für den Kunden sein wollen. Die Stakeholder werden mithilfe der folgenden Tabellen und Abbildungen (\cref{tab:Charakterisierung}, \cref{tab:Maßnahmen}, \cref{fig:Stakeholder}) dargestellt.



\subsubsection*{Charakterisierung der Stakeholder}
\begin{table}[H]
	\centering
	\begin{tabularx}{\textwidth}{|l|l|l|l|X|}
		\hline
		\textbf{Stakeholder} & \textbf{Einfluss} & \textbf{Nähe} & \textbf{Einstellung} & \textbf{Beschreibung} \\
		\hline
		Auftraggeber & groß & nahe & positiv & Geschäftsführer \\
		\hline
		Fahrschulen & groß & mittel & neutral & Die Fahrschulen, welche die Fahrschüler:innen ausbilden. \\
		\hline
		Begleitperson & groß & nahe & positiv & z.B Eltern der Fahrschüler:innen \\
		\hline
		Fahrschüler:innen & groß & nahe & positiv & Die Fahrschüler:innen selbst \\
		\hline
		Medien & gering & mittel & neutral & Alle Medien die Über diese App berichten. \\
		\hline
		Konkurrenz-Apps & groß & nahe & negativ & Konkurrenz-Apps z. Bsp.: ÖAMTC \\
		\hline
	\end{tabularx}
	\caption{Charakterisierung der Stakeholder}
	\label{tab:Charakterisierung}

\end{table}

\subsubsection*{Maßnahmen}
\begin{table}[H]
	\centering
	\begin{tabularx}{\textwidth}{|l|X|}
		\hline
		\textbf{Stakeholder} & \textbf{Maßnahmen} \\
		\hline
		Auftraggeber & Wöchentliche Meetings. Berichte und Fortschritte liefern. \\
		\hline
		Fahrschulen & Überzeugen, dass das Protokoll weniger leicht verfälscht werden kann und die App die Fahrschüler:innen unterstützt. \\
		\hline
		Begleitperson & Protokollierung mittels App soll eine Erleichterung sein und Schutz vor Datenverlust bieten. \\
		\hline
		Fahrschüler:innen & Positive Werbung, benutzerfreundliches und praktisches Design der App, sowie werbungsfrei. \\
		\hline
		Medien & Durch Erfolg und hohe Userzahlen, sowie gutes Marketing / auf App aufmerksam machen. \\
		\hline
		Konkurrenz-Apps & Faires Konkurrenzverhalten \\
		\hline
	\end{tabularx}
	\caption{Maßnahmenkatalog}
	\label{tab:Maßnahmen}
\end{table}





\subsubsection*{Grafische Darstellung des Umfeldes}
\begin{figure}[H]
	\centering
	\includegraphics[width=15cm]{figures/graf_stakeholder.png}
	\caption{Grafik Stakeholder}
	\label{fig:Stakeholder}
\end{figure}


\subsection{Risikoanalyse}

\begin{itemize}
	\item Risikomatrix
\end{itemize}

\section{Pflichtenheft}

\subsection{Zielbestimmung}
\begin{itemize}
	\item Projektbeschreibung
	\item IST-Zustand
	\item SOLL-Zustand
	\item NICHT-Ziele (Abgrenzungskriterien)
\end{itemize}
\subsection{Produkteinsatz und Umgebung}
\begin{itemize}
	\item Anwendungsgebiet
	\item Zielgruppen
	\item Betriebsbedingungen
	\item Hard-/Softwareumgebung
\end{itemize}
\subsection{Funktionalitäten}

	Es gibt grundsätzlich 2 Arten von Anforderungen. Die welche in der App sein müssen und die welche in der App als zusätzlich sein können. Die Muss Anforderungen müssen natürlich vollständig umgesetzt werden. Weiters gibt es die FURPS - Anforderungen. Das sind funktionale Anforderungen und die nicht für die Funktion der App notwendigen Anforderungen wie Benutzerfreundlichkeit, Zuverlässigkeit, Leistung und Support.
	\subsubsection{MUSS Anforderungen}
	\emph{Funktional}
		\begin{itemize}
			\item zuverlässiges und fehlerfreies Tracking der Strecke 
			\item zuverlässige Datenbankaktualisierung
			\item zuverlässiger pdf Export
			\item manuelle Einträge unterstützen
			\item Feedback geben

		\end{itemize}
		
	\emph{nicht Funktional}
		\begin{itemize}
			\item ein ansprechendes Userinterface haben
			\item eine flüssige Menüführung haben
			\item erweiterbar sein
		\end{itemize}
	
	\subsubsection{KANN Anforderungen}
	\emph{Funktional}
		\begin{itemize}
			\item Prüfungsfahrten simulieren mit Audioausgabe der Strecke
		\end{itemize}

	\emph{nicht Funktional}
	\begin{itemize}
		\item verschiedene Designs unterstützen 
		\item in mehreren Sprachen zur Verfügung gestellt werden
	\end{itemize}

\subsection{Testszenarien und Testfälle}
\begin{itemize}
	\item Beschreibung der Testmethodik
	\item Testfall 1
	\item Testfall 2
	\item \ldots
\end{itemize}
\subsection{Liefervereinbarung}
\begin{itemize}
	\item Lieferumfang
	\item Modus
	\item Verteilung(Deployment)
\end{itemize}

\section{Problemanalyse}
In diesen Abschnitt werden mittels UML Diagrammen und Beschreibungen die Funktionalitäten beschrieben. Auch das Userinterface wird dargestellt. Eine gründliche Untersuchung dieser Funktionellen und nicht Funktionellen Anforderungen wird dazu beitragen die Wünsche der Benutzer zu definieren um die App nutzerfreundlich und nach Wunsch des Kunden zu designen. Das soll der Grundbaustein für unsere weitere App Entwicklung sein.

\subsection{USE-Case-Analyse}

USE-Cases, auch Anwendungsfälle genannt, sind eine wichtige Komponente in einer Diplomarbeit. Sie beschreiben, wie ein bestimmtes System oder eine Software in der Praxis eingesetzt wird. Dabei unterscheidet man zwischen benutzerbasierten und ereignisbasierten USE-Cases.

\subsection{Userstories}
\begin{itemize}
	\item Ich als Begleitperson möchte die gefahrenen Routen automatisch aufgezeichnet bekommen.
	\item Ich als Begleitperson möchte eine Fahrt auch manuell hinzufügen können
	\item Ich als Nutzer der App möchte eine Liste der unternommenen Fahrten haben
	\item Ich als Begleitperson möchte die gefahrenen Routen als PDF-Dokument drucken können.
	\item Ich als Fahrschüler:in möchte durch den digitalen Fahrlehrer auf die praktische Fahrprüfung vorbereitet werden.
	\item Als Fahrlehrer:in möchte ich die Fahrdaten auswerten können
	\item Als Benutzer dieser App möchte ich meine Daten sichern können
	\item Ich als Fahrlehrer:in möchte Einsicht auf die gefahrenen Strecken und den Fortschritt meiner Fahrschüler:in haben.
	\item Ich als Fahrschule möchte die Validität der gefahrenen Strecken der Fahrschüler:innen sicherstellen können.
	\item Ich als Fahrschüler:in möchte, wenn es einen Netz-ausfall oder einen GPS-ausfall gibt, trotzdem meine Daten absichern und eine Fahrt weiterführen können.
\end{itemize}

\newpage
Das folgende UML USE-Case-Diagramm (\ref{fig:USE-Case-Diagramm}) zeigt die Userstories nochmal graphisch wie sie in der Anwendung zusammenhängen. 

\begin{figure}[H]
	\centering
	\includegraphics[width=15cm]{figures/usecase_diagramm.png}
	\caption{USE-Case Diagramm}
	\label{fig:USE-Case-Diagramm}
\end{figure}

\newpage
Die nachfolgenden schriftlichen Detailbeschreibungen in Tabelle (\ref{tab:USE-Case1}) zeigen den genauen Ablauf eines Use-Cases. Auf die Auslöser und wer aller beteiligt ist wird beschrieben.
\begin{table}[H]
	\centering
	\begin{tabularx}{\textwidth}{|l|X|}
		\hline
		Use-Case Nummer &  \\
		\hline
		Referenzen &  \\
		\hline
		kurze Beschreibung & Das Fahrtenbuch wird anhand der Rohdaten jedes mal neu erstellt. Diese werden asynchron stückweise über eine Methode geladen um die App zu entlasten. Eine Filterfunktion kann zum Suchen verwendet werden.   \\
		\hline
		Akteure &  Fahrlehrer, Begleitperson, Fahrschüler\\
		\hline
		Auslöser & Der Button Fahrtenbuch wird gedrückt. \\
		\hline
		Vorbedingungen & Mindestens eine Fahrt ist gespeichert. Ansonsten gibt es keine Bedingungen die erfüllt sein müssen um das Fahrtenbuch benutzen zu können. \\
		\hline
		Standardablauf & App starten, Fahrtenbuch öffnen \\
		\hline
	\end{tabularx}
	\caption{Fahrtenbuch öffnen}
	\label{tab:USE-Case1}
\end{table}

\begin{table}[H]
	\centering
	\begin{tabularx}{\textwidth}{|l|X|}
		\hline
		Use-Case Nummer &  \\
		\hline
		Referenzen &  \\
		\hline
		kurze Beschreibung & Eine Fahrt wird automatisch durch GPS-Tracking aufgezeichnet. Im Tunnel bei Signalverlust wird eine Gerade berechnet. Getrackte Fahrten werden für die Analyse verwendet. Es wird nach einer vordefinierten Zeit ein Wegpunkt gespeichert der Längengrad, Breitengrad, Höhe enthält. Damit lässt sich später zum Beispiel eine Heatmap erzeugen. Manuell hinterlegte können dafür nicht verwendet werden.    \\
		\hline
		Akteure &  Fahrlehrer, Begleitperson, Fahrschüler\\
		\hline
		Auslöser & Der Button 'neue Fahrt starten' wird gedrückt. \\
		\hline
		Vorbedingungen & Eine Begleitperson muss mitfahren wird aber softwareseitig nicht kontrolliert. GPS ist eingeschaltet und verfügbar. Tracking wurde akzeptiert. \\
		\hline
		Standardablauf & Im Menü den Button 'neue Fahrt starten' drücken. Anschließend im Formular die Begleitperson, das Kennzeichen und die Wetterverhältnisse auswählen. Beim ersten Start die Meldung Tracking akzeptieren. Danach wird die Fahrt automatisch aufgezeichnet.  \\
		\hline
	\end{tabularx}
	\caption{Fahrtenbuch öffnen}
	\label{tab:USE-Case1}
\end{table}

\subsection{Domain-Class-Modelling}

\begin{itemize}
	\item "Dinge" (Rollen, Einheiten, Geräte, Events etc.) identifizieren, um die es im Projekt geht
	\item ER-Modellierung oder Klassendiagramme
	\item Zustandsdiagramme (zur Darstellung des Lebenszyklus von Domain-Klassen darstellen)
\end{itemize}

\newpage
\subsection{User-Interface-Design}
Wir verwendeten Figma für die Mockups und generellen Designs, Frühzeitige Designs sind in Abbildungen \ref{fig:mockup2}, \ref{fig:mockup1} und \ref{fig:mockup3} zu sehen. Sie zeigen die Erstanmeldung, und die Fahrt-hinzufügen Maske in einem Dark und Light Theme.

\begin{figure}[H]
	\centering
	\includegraphics[width=15cm]{figures/mockup2.png}
	\caption{Fahrt hinzufügen im Dark Theme}
	\label{fig:mockup2}
\end{figure}

\begin{figure}[H]
	\centering
	\includegraphics[width=15cm]{figures/mockup1.png}
	\caption{Anfgangpages Mockup}
	\label{fig:mockup1}
\end{figure}

\begin{figure}[H]
	\centering
	\includegraphics[width=15cm]{figures/mockup3.png}
	\caption{Fahrt hinzufügen im Light Theme}
	\label{fig:mockup3}
\end{figure}

\section{Planung}
\subsection{Projektstrukturplan}
\subsection{Meilensteine}
\subsection{Ablaufplanung}
Gantt-Chart
\subsection{Abnahmekriterien}
\subsection{Pläne zur Evaluierung}
\subsection{Ergänzungen und zu klärende Punkte}

\chapter{Vorstellung des Produktes}
Vorstellung des fertigen Produktes anhand von Screenshots, Bildern, Erklärungen.

\chapter{Eingesetzte Technologien}
\begin{itemize}
	\item Kurzbeschreibung aller Technologien, die verwendet wurden.
	\item Technologien die aus dem Unterricht bekannt sind, nur nennen und deren  Einsatzzweck im Projekt beschreiben, nicht die Technologien selbst.
	\item Technologien die aus dem Unterricht nicht bekannt sind, im Detail beschreiben incl. deren Einsatz im Projekt
	\item Fokus aus eingesetzten Frameworks
\end{itemize}

\chapter{Systementwurf}

\section{Architektur}

\subsection{C4 - Modell}

Beschreibung der Architektur der Software unter Verwendung des C4 Modells: \url{https://c4model.com/}.

Darstellung und Beschreibung der Systemarchitektur.

\begin{itemize}
	\item  statische Zerlegung des Systems in seine physischen Bestandteile (Komponenten, Komponentendiagramm)
	\item (textuelle) Beschreibung des dynamischen Zusammenwirkens aller Komponenten
	\item (textuelle) Beschreibung der Strategie für die Architektur, d. h. wie die Architektur in Statik und Dynamik funktionieren soll.
	\item Verwendung von Referenzarchitekturen bzw. Architekturmustern (als Schablonen, z.B. MVC. Plugin, Pipes and Filters)
	      \begin{itemize}
		      \item MVC
		      \item Schichten
		      \item Pipes
		      \item Request Broker
		      \item Service-Oriented
	      \end{itemize}
\end{itemize}

\subsection{Benutzerschnittstellen}
\begin{itemize}
	\item Design des UIs
	\item Dialoge, Dialogsteuerung, Ergonomie, Gestaltung, Eingabeüberprüfungen
\end{itemize}

\subsection{Datenhaltunskonzept}
\begin{itemize}
	\item Design der Datenbank (ER-Modell)
	\item Design des Zugriffs auf diese Daten (Datenhaltungskonzept)
	\item Caching, Transaktionen
\end{itemize}

\subsection{Konzept für Ausnahmebehandlung}
\begin{itemize}
	\item Systemweite Festlegung, wie mit Exceptions umgegangen wird
	\item Exceptions sind primär aus den Bereichen UI, Persistenz, Workflow-Management
\end{itemize}

\subsection{Sicherheitskonzept}
Beschreibung aller sicherheitsrelevanten Designentscheidungen

\begin{itemize}
	\item Design der Security-Elemente
	\item Design von Safety-Elementen (Fehlertoleranz, Verfügbarkeit etc.)
\end{itemize}

\subsection{Design der Testumgebung}
\begin{itemize}
	\item wie wird getestet (Unit-Testing, Integrationstesting, Systemtests, Akzeptanztests)
	\item Testumgebung, Testprozess, Teststrategie, Testmethoden, Testfälle
\end{itemize}


\subsection{Desing der Ausführungsumgebung}
\begin{itemize}
	\item Deployment (DevOps)
	\item Betrieb (besonders Hoch- und Hertunerfahren der Anwendung)
\end{itemize}

\section{Detailentwurf}

Design jedes einzelnen USE-Cases

\begin{itemize}
	\item Design-Klassendiagramme vom Domain-Klassendiagramm ableiten (incl. detaillierter Darstellung und Verwendung von Vererbungshierarchichen, abstrakten Klassen, Interfaces)
	\item Sequenzdiagramme vom System-Sequenz-Diagramm ableiten
	\item Aktivitätsdiagramme
	\item Detaillierte Zustandsdiagramme für wichtige Klassen
\end{itemize}

Verwendung von CRC-Cards (Class, Responsibilities, Collaboration) für die Klassen
\begin{itemize}
	\item um Verantwortlichkeiten und Zusammenarbeit zwischen Klassen zu definieren und
	\item um auf den Entwurf der Geschäftslogik zu fokussieren
\end{itemize}

Design-Klassen für jeden einzelnen USE-Case können z.B. sein:

\begin{itemize}
	\item UI-Klassen
	\item Data-Access-Klassen
	\item Entity-Klassen (Domain-Klassen)
	\item Controller-Klassen
	\item Business-Logik-Klassen
	\item View-Klassen
\end{itemize}

Optimierung des Entwurfs (Modularisierung, Erweiterbarkeit, Lesbarkeit):

\begin{itemize}
	\item Kopplung optimieren
	\item Kohäsion optimieren
	\item SOLID
	\item Entwurfsmuster einsetzen
\end{itemize}

\chapter{Implementierung}
Detaillierte Beschreibung der Implementierung aller Teilkomponenten der Software entlang der zentralsten Use-Cases:

\begin{itemize}
	\item GUI-Implementierung
	\item Controllerlogik
	\item Geschäftslogik
	\item Datenbankzugriffe
\end{itemize}

Detaillierte Beschreibung der Teststrategie (Testdriven Development):

\begin{itemize}
	\item UNIT-Tests (Funktional)
	\item Integrationstests
\end{itemize}

Zu Codesequenzen:
\begin{itemize}
	\item kurze Codesequenzen direkt im Text (mit Zeilnnummern auf die man in der Beschreibung verweisen kann)
	\item lange Codesequenzen in den Anhang (mit Zeilennummer) und darauf verweisen
\end{itemize}

\chapter{Deployment}
\begin{itemize}
	\item Umsetzung der Ausführungsumgebung
	\item Deployment
	\item DevOps-Thema
\end{itemize}

\chapter{Tests}

\section{Systemtests}
Systemtests aller implementierten Funktionalitäten lt. Pflichtenheft
\begin{itemize}
	\item Beschreibung der Teststrategie
	\item Testfall 1
	\item Testfall 2
	\item Tesfall 3
	\item …
\end{itemize}

\section{Akzeptanztests}

\chapter{Projektevaluation}
siehe Projektmanagement-Unterricht

\chapter{Benutzerhandbuch}
falls im Projekt gefordert

\chapter{Betriebswirtschaftlicher Kontext}
BW-Teil

\chapter{Zusammenfassung}
Mit der Projektumfeldanalyse wurden alle Stakeholder erfasst, die bei einer praktisch ausgelegten App für L17 Fahrschüler:innen als relevant erschienen. Die Positionen der einzelnen Interessensgruppen wurden analysiert und grafisch dargestellt. Weiters wurde ein Maßnahmenkatalog erstellt, um mögliche Methoden zu identifizieren, die das Projektumfeld verbessern sollen. Da dieses Projekt mit der Diplomarbeit am IT-Kolleg zusammenhängt ist der Termin mit Ende Juni für ein Minimum Viable Produkt festgelegt. Dieser Termin ist aber sehr realistisch. Die Anforderungen an diese App sind für eine Smartphone App umfangreich und erfordern eine genaue Auswertung und Prüfung in der Testphase. Dieses Produkt bietet dem/der Fahrschüler:in eine Erleichterung bei der Protokollierung und laufend Feedback über den praktischen Fortschritt dem/der Schüler:in.